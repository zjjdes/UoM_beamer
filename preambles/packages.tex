
\usepackage{graphicx}
\graphicspath{{figures/}}

%\usepackage{mathptmx}      % use Times fonts if available on your TeX system
% insert here the call for the packages your document requires
%\usepackage{latexsym}
\usepackage{amsmath,amsfonts,amscd,amssymb,amsthm,enumerate}
\usepackage{algorithmic}
\usepackage{algorithm}
% \usepackage[caption=false,font=normalsize,labelfont=sf,textfont=sf]{subfig}
%\usepackage{epstopdf}      % used only for presentation
\usepackage{array}
\usepackage{booktabs}
%\setlength{\extrarowheight}{3pt}
\usepackage{balance}
\usepackage{url}
\usepackage{caption}
\captionsetup[figure]{labelformat=empty}
\usepackage{cprotect}

%% From Stanford Beamer tmeplate
\usepackage[utf8]{inputenc}
\usepackage[parfill]{parskip}
\usepackage{subcaption}
\usepackage[]{units}
\usepackage{listings}
\usepackage{multicol}
\usepackage{tcolorbox}
\usepackage{physics}

%% Citation
% \usepackage[natbibapa]{NJDapacite} % ION APA

% For citations in the footnote {
  \usepackage[style=numeric,backend=bibtex,citecounter=true,citetracker=true]{biblatex}
  \addbibresource{references}
  
  \let\svthefootnote\thefootnote
  
  \makeatletter
  % https://tex.stackexchange.com/a/29931/38244
  \DeclareCiteCommand{\cite}
  {\usebibmacro{prenote}}%
  {%
    \ifciteseen{}{%
      \usebibmacro{citeindex}%
      \let\thefootnote\relax%
      \footnotetext{%
        \blx@anchor
        \mkbibbrackets{\usebibmacro{cite}}%
        \setunit{\addnbspace}
          %% Name, title, year
          % \printnames{labelname}%
          % \setunit{\labelnamepunct}
          % \printfield[citetitle]{title}%
          % \newunit
          %% Name, journal, year
          \printnames{author}%
          \setunit{\labelnamepunct}
          \printfield[]{journaltitle}%
          \printfield[]{booktitle}%
          \newunit
          \printfield[]{year}%
      }%
      \let\thefootnote\svthefootnote%
    }%
    \autocite{\thefield{entrykey}}%
  }
  {\addsemicolon\space}
  {\usebibmacro{postnote}}
  \makeatother
  
  \setbeamercolor{footnote}{bg=darkuomblue}
  % \setbeamercolor{footnote mark}{bg=cardinalred}
  \setbeamerfont{footnote}{size=\tiny}
  \setbeamertemplate{footnote}{\insertfootnotetext}
  % }

%%%for color citation%%%%%%%%%%%%%%%%%%%
% \usepackage{xcolor}
% \usepackage{color,hyperref}
% \definecolor{darkblue}{rgb}{0.0,0.0,1.0}
% \definecolor{yell}{rgb}{.8,.9,.0}
% \hypersetup{colorlinks,breaklinks,
%   linkcolor=darkblue,urlcolor=darkblue,
%   anchorcolor=darkblue,citecolor=darkblue}
%%%%%%%%%%%

%%%%%%%%%%
%\usepackage{widetext}
%\usepackage{flushend}
%\usepackage{cuted}
% \usepackage{enumitem}
%%%%%%%%%%

%%%%%%%%%%%%%%%%%%FLOWCHART%%%%%%%%%%%%%%%%
\usepackage{tikz}
\usetikzlibrary{shapes,arrows}
\usetikzlibrary{positioning,shadows,arrows}
\usetikzlibrary{calc}
\usepackage{verbatim}
\definecolor{mycolor}{rgb}{0.2,0.2,0.2}
\definecolor{mygreen}{rgb}{0,0.85,0}
\tikzstyle{input} = [ellipse,draw=black!15, fill=black!15,text width=2em,scale= 0.8, rotate=90, opacity=.4 ,text centered, rounded corners=1mm, minimum height=3.3em,minimum size=1em]
\tikzstyle{input2} = [ellipse,draw=green, fill=black!15,text width=2em,scale= 0.85, rotate=90, opacity=.4 ,text centered, rounded corners=1mm, minimum height=3.4em,minimum size=1.2em]
\tikzstyle{input5} = [ellipse,draw=green, fill=black!15,text width=2em,scale= 0.87, rotate=90, opacity=.4 ,text centered, rounded corners=1mm, minimum height=3.4em,minimum size=1.2em]
\tikzstyle{input4} = [ellipse,draw=green, fill=black!15,text width=2em,scale= 0.87, rotate=90, opacity=.4 ,text centered, rounded corners=1mm, minimum height=3.4em,minimum size=2.4em]
\tikzstyle{input3} = [draw,draw=black!30, fill=white,text width=16.4em,scale= 0.9, text centered, rounded corners=1mm, minimum height=.3em,minimum size=1.1em]
\tikzstyle{input6} = [draw,draw=black, densely dashed,fill=white,text width=17em,scale= 1, opacity=.2, rounded corners=1mm, text centered, minimum height=3em,minimum size=1.1em]
\tikzstyle{line1} = [draw,  color=green, -latex']
\tikzstyle{line2} = [draw,  color=black, -latex']
\tikzstyle{tri} = [draw=black!10, shape border rotate=30, regular polygon, regular polygon sides=3, fill=black!10, scale = .422,node distance=2.5cm, minimum height=2em]
\tikzstyle{tri1} = [draw=black!10, shape border rotate=30, regular polygon, regular polygon sides=3, fill=black!10, scale = .49,node distance=2.5cm, minimum height=2em]
\tikzstyle{trm} = [rectangle,draw=white, fill=white,text width=.3em, scale=.5, text centered, minimum height=1em]
\tikzstyle{crc} = [circle,draw=black!30, fill=white,text width=.3em, scale=.4, text centered, minimum height=1em]
\tikzstyle{trm1} = [rectangle,thin,draw=black!20, fill=white,text width=2.5em, scale=.5, text centered, minimum height=1em]
\tikzstyle{trm2} = [ellipse,draw=green!60, fill=black!10,text width=3.3em, scale=.45 ,text centered, rounded corners=1mm, minimum height=1em]
\tikzstyle{trm3} = [ellipse,draw=green!60, fill=black!10,text width=4em, scale=.45 ,text centered, rounded corners=1mm, minimum height=1em]
\tikzstyle{trm4} = [ellipse,draw=green!60, fill=black!10,text width=5.5em, scale=.41 ,text centered, rounded corners=1mm, minimum height=1em]
\tikzstyle{tra} = [rectangle,draw=white, fill=white,text width=6.1em, scale=.5, text centered, minimum height=1em]
\tikzstyle{tras} = [rectangle,draw=white, fill=white,text width=3em, scale=.5, text centered, minimum height=1em]
\tikzstyle{trat} = [rectangle,draw=white, fill=white,text width=5em, scale=.5,rotate = 90, minimum height=1em]

\tikzstyle{boxisb} = [draw,draw=red!60, fill=white,text width=6em,scale= 0.9, text centered, minimum height=.3em,minimum size=1.1em]
%%%%%%%%%%%%%%%%%%%%%%%%%%%%%%%%%%%%%%%%%%%%%%%%%%%%%%%%%%%%%%%%%%%%%%%

%%%%%%%%%%%%%%%%%%%% Added packages %%%%%%%%%%%%%%%%%%%%%%%%%%%%%%%%%%%
\usepackage{textcomp}
\usepackage{gensymb} % some symbols
% \usepackage{subcaption}
\usepackage{float}
\usepackage{stfloats}
\usepackage{wrapfig}
\usepackage{bm}
  \setcounter{MaxMatrixCols}{30} % maximum columns in a matrix
\usepackage{tabularx}
  \newcolumntype{Y}{>{\centering\arraybackslash}X}

\usepackage{lineno}
% \usepackage{epstopdf}
%%%%%%%%%%%%%%%%%%%%%%%%%%%%%%%%%%%%%%%%%%%%%%%%%%%%%%%%%%%%%%%%%%%%%%%